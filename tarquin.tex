\documentclass{tarquinst}

%\usepackage{showframe}

\begin{document}
	
\begin{changemargin}	
\chapter{What Fraction is Shaded?}	


The aim of these puzzles is to identify the shaded portion of the
whole figure as a fraction of its total area. All of these puzzles are
created using regular shapes. Any additional information required
to solve each puzzle is indicated in the text below the figure.
\end{changemargin}

\newpage 

\noindent 
\questionno{1}\quad 
A parallelogram is inscribed in a square such that two vertices are at
the midpoints of opposite sides as shown.

\noindent
What fraction of the square is shaded?

d

d

d

d

d

d

g

g

d

d

d

d

d

d

d

d

d


d

g

g

d

d

d


\newpage

\noindent
\questionno{2}\quad
The shaded triangle is constructed using the midpoints and a vertex
of an equilateral triangle as shown.
What fraction of the equilateral triangle is shaded?

\newpage 

\begin{changemargin}
\chapter{Solutions and Answers}
\end{changemargin}

\newpage

\noindent 
\answerno{1}\quad 
The eight triangles have the same base (a quarter of a side) and same altitude (side of
the square). So they all have the same area. The shaded area represents one eighth of the
square.
	
\newpage

\noindent 
\answerno{2}\quad 
The rectangle has the same area as these
four grey rhombuses. There are two of
each kind in grey and five of each kind in
the decagon.

	
\end{document}